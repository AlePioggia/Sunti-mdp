% ---------------- RELAZIONE PROGETTO DI PROGRAMMAZIONE AD OGGETTI (OOP) --------
\documentclass[a4paper,12pt]{report}

% ----------------------------- PREAMBLE --------------------------------------- 

\usepackage{lmodern}
\usepackage{alltt, fancyvrb, url}
\usepackage{float}
\usepackage{graphicx}
\usepackage[utf8]{inputenc}
\usepackage{hyperref}
\usepackage{amsmath,amssymb,amsthm}

% Questo commentalo se vuoi scrivere in inglese.
\usepackage[italian]{babel}

\usepackage[italian]{cleveref}

\usepackage{comment}
\usepackage{microtype}
\usepackage{fancyhdr}

\usepackage[scaled=.92]{helvet}
\usepackage[T1]{fontenc}

\usepackage{lscape}

% hyperref settings
\hypersetup{
	colorlinks=true,
	linkcolor=black, %blue
	filecolor=magenta,      
	urlcolor=cyan,
	pdftitle={Sharelatex Example},
	bookmarks=true,
	pdfpagemode=FullScreen,
}

% ----------------------------- PREAMBLE END -----------------------------------

\makeindex

\title{\textbf{FORMULARIO MDP}}
\author{Alessandro Pioggia}

\begin{document}
	
	\makeatletter
	\begin{titlepage}
		\begin{center}
			{\Huge  \@title }\\[3ex] 
			{\large  \@author}\\[3ex] 
			{\large \@date}
		\end{center}
	\end{titlepage}
	\makeatother
	\thispagestyle{empty}
	\newpage
	
	%\maketitle
	
	\tableofcontents
	
	% \input: import the commands from filename.tex to target file.
	
	% \include: does a \clearpage and does an \input.
		
	\newpage
	
	\section{Combinatoria - Lezione 1}
	\subsection{Cos'è la cardinalità di un insieme? Cos'è una corrispondenza biunivoca tra insiemi? Cosa significa insieme numerabile?}
	Supponendo di avere un insieme A con n elementi, n è la cardinalità. 
	Supponendo di avere due insiemi A e B, definiamo una funzione $f(A) : B$, se  è sia suriettiva che iniettiva i due insiemi sono in corrispondenza biunivoca(ovvero è possibile associare a ciascun elemento di un insieme uno ed uno solo elemento dell'altro).
	Un insieme viene detto numerabile se i suoi elementi sono in numero finito o se  possono essere messi in corrispondenza biunivoca con i numeri naturali. Se riesco ad elencare un insieme tramite un elenco infinito è numerabile e viceversa.
	\subsection{Esempio di insieme numerabile e non numerabile}
	Esempio di insiemi numerabili: 
	\begin{itemize}
		\item $P = \{0, 2, 4, 6, ...\}$ $\leftrightarrow$ $\mathbb{N} = \{0, 1, 2, 3, ...\}$ con $f(x) = 2y$
		\item $\mathbb{Z} = \{0, 1, -1, 2, -2, ..\}$ $\leftrightarrow$ $\mathbb{N} = \{0, 1, 2, 3, ...\}$ con $f(z) = 2z - 1$ se $z \ge 0$, $f(z) = -2z$ se $z \leq 0$
	\end{itemize}
	Esempio di insiemi non numerabili:
	\begin{itemize}
		\item L'insieme $\mathbb{R}$
		\item $A$ = \{ Sequenze binarie di lunghezza infinita \}
	\end{itemize}
	\subsection{Che cos'è il prodotto cartesiano?}
	Dati due insiemi A e B, il prodotto cartesiano rappresentato da $A \times B$ non è altro che l'insieme delle coppie ordinate $(a, b)$ con $a \in A$ e $b \in B$ (Dunque (a, b) != (b, a)). Formalmente : $A \times B$ = $\{(a, b) | a \in A, b \in B \}$
	\subsection{Quanti sottoinsiemi ha un insieme con n elementi?}
	Ne ha $2^n$, il risultato lo si ottiene conoscendo l'insieme delle parti(si indica con $p(A)$) e le sue proprietà.
	L'insieme delle parti è appunto il numero di sottoinsiemi di un dato insieme, per giungere alla risposta occorre sfruttare una logica(supponendo di considerare un insieme A con cardinalità n).
	Vogliamo dimostrare che : se $|A| = n$, allora $|p(A)| = 2^n$, dunque ragioniamo in questi termini : 
	\begin{itemize}
		\item fra $p(A)$ e $\{0, 1\}^n$ c'è una corrispondenza biunivoca.
		\item dal principio di uguaglianza abbiamo che, dal momento che esiste una corr. biunivoca,  $|p(A)| = |\{0, 1\}^n|$
		\item dal principio della moltiplicazione otteniamo che $|\{0, 1\}^n| = |\{0,1\}|^n = 2^n$  
	\end{itemize}
	\subsection{Che cos'è una lista?}
	Gli elementi di una potenza cartesiana si dicono liste o sequenze in A e hanno cardinalità n.
	Esempio : \\
	$A^3$ ha una lunghezza delle sequenze o liste pari a 3
	\subsection{Parla dei prodotti condizionati}
	$S \subseteq A \times B$, S è un prodotto condizionato $\leftrightarrow$ posso scegliere la prima coordinata di un elemento di S in n modi e la seconda coordinata, una volta fissata la prima, in m. \\
	Se $S \subseteq A \times B$ è un prodotto condizionato di tipo (n, m) allora $|S| = n \cdot m$. \\
	Dimostrazione : 
	\begin{itemize}
		\item dividiamo $S$ in n sottoinsiemi disgiunti tali che : $\{S_1 \cup S_2 \cup ... \cup S_n\}$
		\item $S_i$ = $\{$ tutti gli elementi di $S$ con prima coordinata $a_i \in A \}$
		\item per definizione abbiamo dunque che $|S_i| = m$
		\item dal momento che i sottoinsiemi considerati sono disgiunti(dal momento che rispettano le condizioni del prodotto condizionato), otteniamo che $S$ = $\{S_1 + S_2 + ... + S_n\}$
		\item $|S|$ = $\{|S_1| + |S_2| + ... + |S_n|\}$ = $\{ m + m + ... + m \}$ = $m \cdot n$
	\end{itemize}
	\subsection{Che cosa sono le disposizioni?}
	Una disposizione di lunghezza k su un insieme di cardinalità n, è una sequenza in cui all'interno non ci sono valori ripetuti. Formalmente : $\{a_1, a_2, ..., a_k\}$ con $a_i \neq a_j$ e $j = i + 1$ è una disposizione di lunghezza k. \\
	Una disposizione lunga k su un insieme di n elementi è esprimibile come prodotto condizionato di tipo : $(n, n-1, n-2, ..., n-k)$, questo perchè una volta che seleziono la prima coordinata, se la seconda deve essere distinta ho n-1 scelte e così via... \\
	Dalla proprietà dei prodotti condizionati dimostrata in precedenza abbiamo che |disposizioni| = $(n \cdot n-1 \cdot n - 2 \cdot ... \cdot n-k)$.
	Se ho k = n $\rightarrow$ |disposizioni| = n!, vengono chiamate permutazioni.
	\subsection{Definisci le combinazioni, illustra inoltre la correlazione che c'è fra esse e le disposizioni, di conseguenza mostra come contare il numero di combinazioni}
	Considerando un insieme A con $\{1, ..., n\}$ elementi, le combinazioni di A sono tutti i suoi sottoinsiemi di cardinalità k. Per contare il numero di combinazioni lunghe k di un insieme con n elementi è possibile sfruttare la correlazione che c'è fra disposizioni e combinazioni. Per ogni combinazione lunga k esistono esattamente k! disposizioni, infatti se mettiamo a funzione disposizioni e combinazioni otteniamo una funzione suriettiva(disposizioni $\rightarrow$ combinazioni). A questo proposito deduciamo che : $|combinazioni| = \dfrac{n_{(k)}}{k!}$.
	\subsection{Come possiamo contare con precisione il numero di anagrammi di una parola?}
	Possiamo sfruttare le combinazione di tipo (a, b, c), con |a| + |b| + |c| = n.
	Gli anagrammi di tipo (a, b, c)  sono le sequenze ordinate in cui appare lo 0 a volte, l'1 b volte, il 2 c volte.
	Inoltre è possibile mettere in biezione le combinazioni di tipo (a, b, c) con le sequenze ternarie ordinate di sottoinsiemi($s_1, s_2, s_3$), in cui $|s_1|$ = a, $|s_2|$ = b, $|s_3|$ = c.(In $s_1$ inserisco gli indice posizionali degli 0 che appaiono, di conseguenza lo faccio anche con $s_2$ e $s_3$).
	Inoltre abbiamo che $s_1 \cup s_2 \cup s_3 = \{0, ..., n\}$
	\subsection{Definisci la funzione ricorsiva di Stifel}
	La funzione ricorsiva di Stifel è la seguente : $\sum_{k = 0}^{n/2}(\binom{n - 1}{k} + \binom{n - 1}{k - 1})$ \\
	Per dimostrarne la correttezza è sufficiente suddividere l'insieme preso in considerazione in due sottoinsiemi, che chiameremo A e B.
	A = { insiemi che contengono n }
	B = { insiemi che non contengono n} \\
	Gli insiemi che non contengono n sono $\binom{n - 1}{k}$, ovvero escludo dalla conta del numero dei sottoinsiemi lo stesso n. Per quanto riguarda A invece, possiamo ottenere il valore $\binom{n - 1}{k - 1}$ perchè considero tutti i sottoinsiemi di cardinalità k - 1 con n escluso, dopodichè n verrà aggiunto a ciascun sottoinsieme.
	\subsection{Parlami dei numeri di Fibonacci}
	I numeri di Fibonacci sono l'insieme delle sequenze di bit che non presentano due 1 consecutivi. \\
	Formula : $f_n = f_{n - 1} \cdot f_{n - 2}$ \\
	Per dimostrare la correttezza della formula è sufficiente suddividere le sequenze fra quelle che: 
	\begin{itemize}
		\item iniziano con 1 : prendo n-2 elementi ed aggiungo all'inizio qualsiasi coppia che non sia (1, 1).  
		\item iniziano con 0 : prendo n-1 elementi ed aggiungo 0 all'inizio.
	\end{itemize}
	C'è anche una formula alternativa per calcolare i numeri di fibonacci : $\sum_{k = 0}^{n/2}\binom{n - k + 1}{k}$ \\
	Dimostrazione : \\
	In pratica in questo caso prendiamo una sequenza e togliamo tutti gli zeri, quindi abbiamo tutti piccoli insiemi che contengono sequenze con soli 1. Funziona perchè so che per rispettare le condizioni, ogni 1 della sequenza deve essere affiancato da uno 0, tranne eventualmente se l'1 ricopre l'ultima posizione, dunque ho n - k zeri - 1(che sta in fondo).
	\subsection{Raccontami il principio di inclusione-esclusione}
	Il principio di inclusione-esclusione permette di calcolare la cardinalità di un insieme, espresso come unione di n insiemi finiti. \\
	In formule(Si ricorda che I	è l'insieme di sottoinsiemi di un insieme con {1, ..., n} elementi) : 
	\begin{itemize}
		\item $|A_1 \cup A_2 \cup ... \cup A_n| = \sum_{I \in {0, ..., n}}(-1)^{|I|-1} \cdot \prod_{i \in I}|A_i| $
	\end{itemize}
	Detto ciò sappiamo che molto spesso viene sfruttata la formula che ci permette di arrivare al complementare. Per fare ciò è opportuno conoscere l'insieme universo U, che contiene tutti gli elementi e tutti gli insiemi esistenti(quindi anche sè stesso e l'insieme vuoto). Prima di enunciare la formula è importante sapere che : $|U| = |\prod_{i \in 0}A_i|$. \\ \\
	$|(A_1 \cup A_2 \cup ... \cup A_n|)^c = |U| -  \sum_{I \in {0, ..., n}}(-1)^{|I|-1} \cdot \prod_{i \in I}|A_i|$ \\\\
	= $|\prod_{i \in 0}A_i| - \sum_{0 \neq I \in {0, ..., n}}(-1)^{|I|-1} \cdot \prod_{i \in I}|A_i|$  \\\\
	= $|\prod_{i \in 0}A_i| + \sum_{0 \neq I \in {0, ..., n}}(-1)^{|I|} \cdot \prod_{i \in I}|A_i|$ \\\\
	= $\sum_{I \in {0, ..., n}}(-1)^{|I|} \cdot \prod_{i \in I}|A_i|$
	\subsection{Come faccio a contare il numero di funzioni iniettive?} 
	RECAP : Una funzione iniettiva è definita tale se elementi distinti del dominio hanno immagini distinte. \\
	Supponendo di avere una funzione iniettiva $f:A \rightarrow B$ con m $\leq$ n(poniamo |A| = m e |B| = n), il numero di funzioni iniettive è $n_{m}$ infatti, piccolo appunto, non è altro che un prodotto condizionato di tipo (n - 1, n-2,  ..., n-m).
	\subsection{Come faccio a contare il numero di funzioni suriettive?}
	RECAP : Una funzione suriettiva è definita tale se ogni elemento del codominio ha come immagine almeno un elemento del dominio.
	Supponendo di avere una funzione suriettiva $f:A \rightarrow B$ con m $\geq$ n, (poniamo ancora |A|=m e |B|=n), il numero di funzioni suriettive è possibile calcolarlo sfruttando il principio di inclusione-esclusione.
	Supponiamo di avere un insieme A composto da n sottoinsiemi, ovvero A = $\{A_1 \cup A_2 \cup ... \cup A_n\}$ \\ in cui definiamo $A_i$ = \{ tutte le funzioni in cui l'elemento i $\notin$ Imf \}, la sua cardinalità è : $|A_i| = (n - 1)^m$, stessa regola la usiamo per l'intersezione, ovvero $|A_i \cap A_j| = (n - 2)^m$ e dunque generalizzando avremo che : \\ $|\prod_{I \in {0...n}}A_i| = (n - |I|)^m$ In seguito a questa deduzione osserviamo che se, la nostra funzione appartiene anche solo ad uno degli $|A_i|$ insiemi, non può essere definita suriettiva, dal momento che ogni elemento del codominio deve avere come immagine almeno un elemento del dominio. Quindi per contare il numero di funzioni suriettive occorre calcolare  il complementare dell'insieme che abbiamo preso in considerazione prima, ovvero $|(A_1 \cup A_2 \cup ... \cup A_n)^c|$, sfruttando il principio di inclusione-esclusione. \\ \\
	 $|(A_1 \cup A_2 \cup ... \cup A_n)^c| = |\prod_{I \in {0...n}}A_i| - \sum_{0 \neq i \in I}(-1)^{(|I| - 1)}|\prod_{I \in {0...n}}A_i|$ \\ \\
	 = 
	 = $\sum_{i \in I}(-1)^{(|I|)} \cdot (n - |I|)^m$ \\ \\
	 = $\sum_{k = 0}^{n}(-1)^k \cdot \binom{n}{k} \cdot (n - k)^m$
	 
	 
	
\end{document}
